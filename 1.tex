\documentclass{article}
\usepackage{ctex}
\usepackage{graphicx}
\usepackage{subcaption}
\usepackage{listings}
\title{实验复现报告}
\date{\today}
\author{XIA}

\begin{document}

\maketitle

https://github.com/Game-learning/recurrent

\tableofcontents

\newpage

\section{图像增强}


\begin{figure}[h!]
    \centering
    \begin{subfigure}{0.4\textwidth}
        \centering
        \includegraphics[width=\linewidth]{photo1.jpg}
        \caption{}
        \label{fig:subfig1}
    \end{subfigure}
    \hfill
    \begin{subfigure}{0.4\textwidth}
        \centering  
        \includegraphics[width=\linewidth]{cc.png}
        \caption{}
        \label{fig:subfig2}
    \end{subfigure}

\end{figure}

可以将一张模糊的照片增强为一张清晰的照片,提高图像的质量。

\section{文本到语音转换}

输入文本:

text = "Hello, this is a test of speech synthesis using Tacotron2."

\begin{figure}[h!]
        \centering
        \includegraphics[width=\linewidth]{output_4_0.png}

\end{figure}

可以将一段英文读成一个句子的语音。将中文文本转化为语音方面有待学习与探索。

\newpage


\section{文本到图片转换}

输入文本:


Astronaut riding a horse.

\begin{figure}[h!]
    \centering
    \includegraphics[width=\linewidth]{123.png}

\end{figure}

可以根据英文提示词,生成一张对应的图片。


\newpage


\section{光学字符识别}




\begin{figure}[h!]
    \centering
    \begin{subfigure}{0.4\textwidth}
        \centering
        \includegraphics[width=\linewidth]{cc2.jpg}
        \caption{}
        \label{fig:subfig3}
    \end{subfigure}
    \hfill
    \begin{subfigure}{0.4\textwidth}
        \centering  
        \includegraphics[width=\linewidth]{wolf_gray.jpg}
        \caption{}
        \label{fig:subfig4}
    \end{subfigure}

\end{figure}

图片\ref{fig:subfig3}描述: a man and woman standing on a cliff overlooking a lake

图片\ref{fig:subfig4}描述: a wolf is walking through a hole in the snow

可以识别图片,生成对图片的描述。

\section{内容审核}

输入文本:

"Those people are all garbage and don't deserve to live,You idiot, you can't do anything well."

\begin{table}[h!]
    \centering
    \begin{tabular}{l c c c c c c c }
    \hline

    类别 & toxic & severe\_toxic & obscene & threat & insult & identity\_hate \\   \hline

    是否存在 & True & False & False & False & True & False \\
 
    \hline
    \end{tabular}
    \caption{内容判定情况}
    \end{table}

可以根据文本内容,判断文本情感。

\newpage
\section{Python代码}

\begin{lstlisting}[language=Python]
    ######推荐使用虚拟环境,安装所需的包,避免出错。

#图像增强(cmd)-------------------------------------------------------------------
git clone https://github.com/xinntao/Real-ESRGAN.git
cd Real-ESRGAN
# 安装 basicsr - https://github.com/xinntao/BasicSR
# 我们使用BasicSR来训练以及推断
pip install basicsr
# facexlib和gfpgan是用来增强人脸的
pip install facexlib
pip install gfpgan
pip install -r requirements.txt
python setup.py develop

wget https://github.com/xinntao/Real-ESRGAN/releases/download/v0.1.0/RealESRGAN_x4plus.pth -P weights

#python inference_realesrgan.py -n RealESRGAN_x4plus -i inputs --face_enhance

python inference_realesrgan.py -n RealESRGAN_x4plus -i photo1.jpg --face_enhance -o output_folder



#文本到语音转换-----------------------------------------------------------------
import torch
import torchaudio
import matplotlib.pyplot as plt
from torchaudio.models import Tacotron2
from torchaudio.pipelines import TACOTRON2_WAVERNN_PHONE_LJSPEECH

# 加载 Tacotron2 模型和 WaveRNN 声码器
bundle = TACOTRON2_WAVERNN_PHONE_LJSPEECH
processor = bundle.get_text_processor()
tacotron2 = bundle.get_tacotron2()
vocoder = bundle.get_vocoder()

# 将模型设置为评估模式
tacotron2.eval()
vocoder.eval()

# 输入文本
text = "Hello, this is a test of speech synthesis using Tacotron2."

# 处理文本
with torch.inference_mode():
    processed, lengths = processor(text)

    # 生成梅尔频谱
    mel_spec, mel_lengths, _ = tacotron2.infer(processed, lengths)

    # 使用声码器生成波形
    waveforms, lengths = vocoder(mel_spec, mel_lengths)

# 可视化波形
plt.figure(figsize=(10, 4))
plt.plot(waveforms[0].cpu().numpy())
plt.title("Generated Waveform")
plt.xlabel("Sample")
plt.ylabel("Amplitude")
plt.show()

# 保存音频文件
output_path = "output.wav"
torchaudio.save(output_path, waveforms.cpu(), sample_rate=22050)
print(f"Audio has been saved to {output_path}")



#文本到图片转换-----------------------------------------------------------------

import requests
import io
from PIL import Image

#API_URL = ""  # 你的API地址
#headers = {}  # 你的请求头
def query(payload):
    response = requests.post(API_URL, headers=headers, json=payload)
    return response.content

image_bytes = query({
    "inputs": "Astronaut riding a horse",     # 输入英文提示词
})

image = Image.open(io.BytesIO(image_bytes))



#光学字符识别-------------------------------------------------------------------------------------

from transformers import BlipProcessor, BlipForConditionalGeneration
from PIL import Image
import requests

# 加载BLIP模型和处理器
processor = BlipProcessor.from_pretrained("Salesforce/blip-image-captioning-base")
model = BlipForConditionalGeneration.from_pretrained("Salesforce/blip-image-captioning-base")

# 读取图片
image_path = 'cc2.jpg'  # 替换为你的图片路径
image = Image.open(image_path)

# 预处理图片
inputs = processor(images=image, return_tensors="pt")

# 生成描述
out = model.generate(**inputs)

# 解码并输出描述
description = processor.decode(out[0], skip_special_tokens=True)
print("图片描述:", description)

#内容审核-----------------------------------------------------------------------------
import pandas as pd
from sklearn.model_selection import train_test_split
from sklearn.feature_extraction.text import TfidfVectorizer
from sklearn.linear_model import LogisticRegression
from sklearn.multioutput import MultiOutputClassifier
from sklearn.metrics import classification_report

# 加载数据集
train_df = pd.read_csv('train.csv')
test_df = pd.read_csv('test.csv')
test_labels_df = pd.read_csv('test_labels.csv')
sample_submission_df = pd.read_csv('sample_submission.csv')

# 随机提取100000个样本
train_sample = train_df.sample(n=100000, random_state=12)
test_sample = test_df.sample(n=100000, random_state=12)

# 注意,test_labels中的ID需要与test_sample中的ID匹配
test_labels_sample = test_labels_df[test_labels_df['id'].isin(test_sample['id'])]

# sample_submission中的ID也需要与test_sample中的ID匹配
sample_submission_sample = sample_submission_df[sample_submission_df['id'].isin(test_sample['id'])]

# 保存样本数据
train_sample.to_csv('train_sample.csv', index=False)
test_sample.to_csv('test_sample.csv', index=False)
test_labels_sample.to_csv('test_labels_sample.csv', index=False)
sample_submission_sample.to_csv('sample_submission_sample.csv', index=False)




# 加载样本数据集
train_sample = pd.read_csv('train_sample.csv')

# 提取特征和标签
X = train_sample['comment_text']
y = train_sample[['toxic', 'severe_toxic', 'obscene', 'threat', 'insult', 'identity_hate']]

# 分割数据集
X_train, X_val, y_train, y_val = train_test_split(X, y, test_size=0.2, random_state=42)

# 特征提取
vectorizer = TfidfVectorizer(max_features=10000)
X_train_tfidf = vectorizer.fit_transform(X_train)
X_val_tfidf = vectorizer.transform(X_val)

# 训练多标签分类模型
model = MultiOutputClassifier(LogisticRegression(max_iter=1000))
model.fit(X_train_tfidf, y_train)

# 预测和评估
y_pred = model.predict(X_val_tfidf)
print(classification_report(y_val, y_pred, target_names=y.columns))

def predict_labels(comment):
    comment_tfidf = vectorizer.transform([comment])
    prediction = model.predict(comment_tfidf)
    labels = y.columns
    return {label: bool(pred) for label, pred in zip(labels, prediction[0])}

    # 示例句子
    comment = "Those people are all garbage and don't deserve to live"
    prediction = predict_labels(comment)
    print(prediction)
    # 示例句子
    comment = "You idiot, you can't do anything well."
    prediction = predict_labels(comment)
    print(prediction)
    

\end{lstlisting}

\end{document}